% \todo[inline]{Move the introductions of both the papers here, and remove the paper specific introductions from the chapters.}


The moduli space $\mg(\no_g)$ of compact \emph{non-orientable} hyperbolic surfaces of genus $g$ is conjectured to have similarities to infinite volume geometrically finite manifolds (in a manner similar to how moduli spaces of compact orientable surfaces have properties similar to finite volume hyperbolic manifolds).
The main results suggesting the analogy between moduli spaces of non-orientable surfaces and infinite volume geometrically finite manifolds are due to Norbury and Gendulphe.

\begin{itemize}
\item The $\mg(\no_g)$ has infinite Teichm\"uller volume \cite[Theorem 17.1]{gendulphe2017whats}.
  While the associated Teichm\"uller space does not have a Weil-Petersson volume form, it has an analogous volume form with respect to which the moduli space has infinite volume as well (see \cite{norbury2008lengths}).
\item The action of the mapping class group $\mcg(\no_g)$ on the Thurston boundary is not minimal (Proposition 8.9 in \cite{gendulphe2017whats}).
\item The Teichm\"uller geodesic flow is not topologically transitive, and thus not ergodic with respect to any Borel measure with full support \cite[Proposition 17.5]{gendulphe2017whats}.
\item There exists an $\mcg(\no_g)$-equivariant finite covolume deformation retract of $\teich(\no_g)$.
\end{itemize}

We extend this analogy further, by showing that the limit set of $\mcg(\no_g)$ is contained in the complement of a full measure dense open set.
\begingroup
\def\thetheorem{\ref{cor:geolimset}}
\begin{theorem}
  The limit set of $\mcg(\no_g)$ is contained in the complement of $\pmf^-(\no_g)$.
\end{theorem}
\addtocounter{theorem}{-1}
\endgroup
Here $\pmf^-(\no_g)$ is the set of all projective measured foliations that have one-sided compact leaf.
The fact that such foliations form a full measure dense open subset is classical, due to Danthony-Nogueira (see \cite{ASENS_1990_4_23_3_469_0}).
This is analogous to limit sets of infinite volume geometrically finite groups, where the complement of the limit set is a full measure open set as well.

In \cite{gendulphe2017whats}, Gendulphe constructed a retract of $\teich(\no_g)$ to $\systole(\no_g)$, the set of points in the Teichm\"uller space that have no one-sided curves shorter than $\varepsilon$, and showed that it has finite covolume.
They also asked the following question about $\systole(\no_g)$.
\begin{unquestion}[Question 19.1 of \cite{gendulphe2017whats}]
  Is $\systole(\no_g)$ quasi-convex with respect to the Teichmüller metric?
\end{unquestion}
We show that ${\systole(\no_g)}$ is not quasi-convex, answering the above question.
\begingroup
\def\thetheorem{\ref{thm:qc-fail}}
\begin{theorem}
  For all $\varepsilon > 0$, and all $D > 0$, there exists a Teichm\"uller geodesic segment whose endpoints lie in ${\systole(\no_g)}$ such that some point in the interior of the geodesic is more than distance $D$ from $\systole(\no_g)$.
\end{theorem}
\addtocounter{theorem}{-1}
\endgroup

Since $\systole(\no_g)$ is an $\mcg(\no_g)$-invariant subset of $\teich(\no_g)$, the intersection of its closure with the boundary must also be $\mcg(\no_g)$-invariant, and therefore contain the limit set of $\mcg(\no_g)$.
This suggests that if we want long geodesic segments that start and end in $\systole(\no_g)$, we must look for Teichmüller geodesics that have their expanding and contracting foliations in the limit set.
Conjecture 9.1 of \cite{gendulphe2017whats} states that the limit set should exactly be the complement of $\pmf^-(\no_g)$, the set of projective measured foliations that do not contain any one-sided leaves (denoted $\pmf^+(\no_g)$).
We prove a result that is slightly weaker than the conjecture.
\begingroup
\def\thetheorem{\ref{thm:rational-approximation}}
\begin{theorem}
  A foliation $\lambda \in \pmf^+(\no_g)$ is in the limit set of $\mcg(\no_g)$ if all the minimal components $\lambda_j$ of $\lambda$ satisfy one of the following criteria.
  \begin{enumerate}[(i)]
  \item $\lambda_j$ is periodic.
  \item $\lambda_j$ is ergodic and orientable, i.e. all leaves exiting one side of a transverse arc always come back from the other side.
  \item $\lambda_j$ is uniquely ergodic.
  \end{enumerate}
  Furthermore, if $\lambda_j$ is minimal, but not uniquely ergodic, there exists some other foliation $\lambda_j^{\prime}$ supported on the same topological foliation as $\lambda_j$ which is in the limit set.
\end{theorem}
\addtocounter{theorem}{-1}
\endgroup

Combining Theorem \ref{thm:rational-approximation} with a result of \textcite[Proposition 1]{lenzhen2010criteria}, a complete description of the limit set can be obtained.
\begin{theorem}[Also proven independently by \cite{erlandsson2023mapping}]
  The limit set of $\mcg(\no_g)$ in $\pmf(\no_g)$ is $\pmf^+(\no_g)$.
\end{theorem}

With this description of the limit set, we prove \autoref{thm:qc-fail} by constructing a family of Teichmüller geodesics whose expanding and contracting foliations are of the kind described by \autoref{thm:rational-approximation}, and showing that some point in the interior of the geodesic segment is arbitrarily far from $\systole(\no_g)$.

By understanding of the failure of convexity of $\systole(\no_g)$, we can try to strengthen the analogy to geometrically finite manifolds by constructing a family of Patterson-Sullivan measures on the limit set.
However, we still need a convex core, if we want a good analogy with geometrically finite manifolds: we show that the failure of convexity of $\systole(\no_g)$ is not a serious obstruction to understanding geodesic segments whose endpoints lie in $\systole(\no_g)$.


\begingroup
\def\thetheorem{\ref{thm:weak-convexity}}
\begin{theorem}
For any $\vepd > 0$, there exists constants $\vept^{\prime}$ and $c$, such that any geodesic segment $\gamma$, whose length is more than $c$, with endpoints in $\systole(\no_g)$, for $0 < \vept < \vept^{\prime}$, can be homotoped to a segment relative to endpoints to lie entirely within $\systole(\no_g)$, such that the length of the homotoped segment $\gamma^{\prime}$ satisfies the following inequality.
  \begin{align*}
    \ell(\gamma^{\prime}) \leq \ell(\gamma) \cdot (1 + \vepd)
  \end{align*}
\end{theorem}
\addtocounter{theorem}{-1}
\endgroup
% \todo[inline]{Should not have $\vept$ and constant $t$ in the same theorem statement.}

Theorem \ref{thm:weak-convexity} shows that $\systole(\no_g)$, despite not being convex, almost behaves like the convex core of $\teich(\no_g)$: it is a metric subset of $\teich(\no_g)$ (with respect to the induced path metric) which is distorted by an arbitrarily small amount.
We call $\systole(\no_g)$ the \emph{weak convex core} of $\teich(\no_g)$, and focus our attention on this subspace as a metric space, where the metric is the induced path metric.
If we restrict our attention to the cotangent directions in $\teich(\no_g)$ along which the geodesic flow does not eventually leave $\systole(\no_g)$, we can use those cotangent directions to define a geodesic flow for $\systole(\no_g)$.
We call this collection of restricted directions the restricted cotangent bundle over $\systole(\no_g)$.

% If we restrict our attention to the cotangent directions in which the Teichmüller geodesic enters $\systole(\no_g)$ for arbitrarily large times\todo{Is this clear?}, we can define a \emph{slowed-down} geodesic flow on those tangent directions using the geodesic flow in $\teich(\no_g)$.
% This new geodesic flow projects down to $\systole(\no_g)$, and is only slower than the usual Teichmüller geodesic flow by a factor of at most $1 + \vepd$.
% \todo{Might wanna remove it? Or say, ``if we restrict our attention to tangent directions that do not eventually leave systole, we can define a geodesic flow for this new space.''}

In light of this, we restrict our attention to $\systole(\no_g)$, and the $\mcg(\no_g)$ action on $\systole(\no_g)$.
Since the action of $\mcg(\no_g)$ on $\systole(\no_g)$ is finite $\nu_N$-covolume (but not cocompact), one might try to prove that the action is
% \emph{analogous} to
{like}
the action of lattices in $\mathrm{SL}_2(\mathbb{R})$ on $\mathbb{H}$.
However, the results on lattices (and Teichmüller spaces of orientable surfaces) rely on having a measure preserving $\mathrm{SL}_2(\mathbb{R})$ action on the unit tangent bundle (respectively on the moduli space of quadratic differentials), and use the interplay between the geodesic flow and the horocycle flow.

For non-orientable surfaces, we do not have an analog of the horocycle flow on the space of quadratic differentials, so we cannot hope to directly import the techniques from the orientable case.
However, \textcite{10.1093/imrn/rny001} introduced a notion of \emph{statistically convex-cocompact action}, which can replace the notion of a lattice-like action for our setting.
In the setting of $\systole(\no_g)$, proving statistical convex-cocompactness is equivalent to proving that geodesic segments between $\mcg(\no_g)$ orbit points in $\systole(\no_g)$ enter the thin part (i.e. the region in $\systole(\no_g)$ where some two-sided curve is short) with exponentially low probabilities.

Our next result is that this holds for the $\mcg(\no_g)$ action on $\systole(\no_g)$.
\begin{theorem}[Corollary of Theorems \ref{thm:entropy-equality-implies-scc} and \ref{thm:entropy-equality}]
  \label{thm:statistical-convex-cocompactness}
  The action of $\mcg(\no_g)$ on $\systole(\no_g)$ is statistically convex-cocompact.
\end{theorem}
% \todo{Is stating that statistical convexity follows as a corollary of two theorems okay?}

\textcite{gekhtman2023dynamics}, and \textcite{CGTY} (in upcoming work) prove the following theorems about statistically convex cocompact actions of subgroups of mapping class groups on Teichmüller space.

\begin{theorem}[To appear in \textcite{CGTY}]
  \label{thm:bms-finite}
  Let $\Gamma$ be a subgroup of the mapping class group $\mcg(\os_g)$ whose action on $\teich(\no_g)$ is statistically convex-cocompact.
  Then the Bowen-Margulis measure $\mu_{\mathrm{BMS}}$ on the unit cotangent bundle of $\teich(\os_g)/\Gamma$ is finite.
\end{theorem}

\begin{theorem}[Theorem 4.9 of \cite{gekhtman2023dynamics}]
  \label{thm:mixing-bms}
  If $\Gamma$ is a subgroup of the mapping class group such that the associated Bowen-Margulis measure is finite, then the geodesic flow is mixing with respect to $\mu_{\mathrm{BMS}}$.
\end{theorem}

As a corollary of the Theorem \ref{thm:mixing-bms}, we get the following results about the Patterson-Sullivan measures and the action of $\Gamma$ on the limit set.
\begin{corollary}
  \label{cor:niceness}
  If $\Gamma$ is a subgroup of $\mcg(\os_g)$ such that the Bowen-Margulis measure associated to $\Gamma$ is finite, then the following results hold.
  \begin{itemize}
  \item There exists a unique family of conformal densities $\left\{ \mu_x \right\}_{x \in \teich(\os_g)}$ on the limit set of $\Gamma$, given by the Patterson-Sullivan construction.
  \item The action of $\Gamma$ on the limit set is ergodic.
  \end{itemize}
\end{corollary}

We believe Theorem \ref{thm:bms-finite} also works when we consider the $\mcg(\no_g)$ action on $\systole(\no_g)$ instead of $\teich(\os_{g-1})$ (i.e\. the Teichmüller space of the orientation double cover), since the proof of Theorem \ref{thm:bms-finite} relies on considering the limit set of the group, which is the same for the action on $\systole(\no_g)$ as it is for the action on the entire Teichmüller space (we proved this in \cite[Theorem 4.2]{limitsetkhan}).

Combining Theorem \ref{thm:statistical-convex-cocompactness} with Theorems \ref{thm:bms-finite} and \ref{thm:mixing-bms} lets us construct Patterson-Sullivan measures (with good dynamical properties) on the limit set, and a finite Bowen-Margulis measure on the unit cotangent bundle.
% In particular, \textcite{CGTY} show that the construction of Patterson-Sullivan measures for statistically convex-cocompact subgroups of the mapping class groups leads to a finite geodesic flow invariant measure (i.e. the Bowen-Margulis measure) on the tangent bundle, and \textcite{gekhtman2023dynamics} show that finite Bowen-Margulis measure implies mixing of the geodesic flow in this setting.
In particular, letting $\Gamma = \mcg(\no_g) < \mcg(\os_{g-1})$ gives us a version of Theorem \ref{thm:mixing-bms} and Corollary \ref{cor:niceness} for $\mcg(\no_g)$.

\begin{corollary}
  There exists a unique family of conformal densities $\left\{ \mu_x \right\}_{x \in \teich(\no_g)}$ on the limit set $\pml^+(\no_g)$, given by the Patterson-Sullivan construction.
  Furthermore, the action of $\mcg(\no_g)$ on $\pml^+(\no_g)$ is ergodic, and the geodesic flow on the restricted cotangent bundle over $\systole(\no_g)$ is mixing with respect to the Bowen-Margulis measure.
\end{corollary}


\subsection*{Why we care about the limit set of $\mcg(\no_g)$ and Patterson-Sullivan measures}

\subsubsection*{Counting problems}

Understanding the dynamics of the geodesic flow over the moduli space of \emph{orientable surfaces} has led to solutions for two counting problems: one on the moduli space of hyperbolic surfaces, and one on hyperbolic surfaces themselves.

\begin{enumerate}[(i)]
\item Counting closed curves in moduli space: Via techniques originally introduced to Margulis in his thesis \cite{margulis2004some}, one can reduce counting closed curves, which are conjugacy classes of mapping class group orbit points, to understanding the geodesic flow over the moduli space.
  The number of closed curves of length at most $R$, which we denote by $N(R)$ has the following asymptotics (see \cite{eskinmirzakhani}).
  \begin{align}
    \label{eq:counting-closed}
    N(R) \sim \frac{\exp(hR)}{hR}
  \end{align}
  Here, the symbol $\sim$ means that the ratio of the two quantities approaches a positive constant as $R$ goes to $\infty$, and $h$ is the volume growth entropy of $\teich(\os_{g})$, which is $6g-6$.
\item Counting \emph{simple} closed curves on orientable hyperbolic surfaces: \textcite{mirzakhani2008growth} proved that the counting function $M(R)$ that counts \emph{simple} closed curves satisfies a polynomial asymptotic.
  \begin{align}
    \label{eq:counting-simple-closed}
    M(R) \sim R^{h}
  \end{align}
  Here, $h$ is again the volume growth entropy, i.e. $6g-6$.
  This count also led to an explicit computation of the volumes of moduli spaces of orientable hyperbolic surfaces with boundary, as well as the calculation of expected values for various geometric properties of Weil-Petersson random hyperbolic surfaces.
\end{enumerate}

For non-orientable surfaces, the counting function does not behave like the orientable version.
% \textcite{gendulphe2017whats} showed that the counting function for closed curves in $\no_g$ is $o(\exp((3g-6)R))$, and the counting function for simple closed curves in $o(R^{3g-6})$, where $3g-6$ is the dimension of $\teich(\no_g)$.
% This raises the question of whether there is an exponent $h < 3g-6$ for which the non-orientable versions of \eqref{eq:counting-closed} and \eqref{eq:counting-simple-closed} continue to hold.
\textcite{gendulphe2017whats} showed that the counting function $N_{\mathrm{no}}(R)$ and $M_{\mathrm{no}}(R)$, which is the versions of the functions $N(R)$ and $M(R)$ for non-orientable surfaces satisfy the following asymptotic.
\begin{align*}
  N_{\mathrm{no}}(R) &= o\left( \frac{\exp((3g-6)R)}{(3g-6)R} \right) \\
  M_{\mathrm{no}}(R) &= o(R^{3g-6})
\end{align*}
These asymptotics raise the question of whether there is an exponent $h < 3g-6$ for which the non-orientable versions of \eqref{eq:counting-closed} and \eqref{eq:counting-simple-closed} continue to hold.

By establishing that the action of $\mcg(\no_g)$ on $\teich(\no_g)$ is statistically convex-cocompact, we can use the results of Coulon, Gekhtman, Ma, Tapie, and Yang to count lattice points and their conjugacy classes to obtain a non-orientable version of \eqref{eq:counting-closed} where the role of $h$ is played by the critical exponent for the group action of $\mcg(\no_g)$ with respect to the Teichmüller metric.

To explain how ergodicity of the $\mcg(\no_g)$ action on $\pml^+(\no_g)$ might help count simple closed curves on $\no_g$, we outline Mirzakhani's original proof of the fact for orientable surfaces (see \cite{mirzakhani2008growth} for the original proof, and \cite{2022arXiv220204156A} for a gentler exposition).

\begin{proof}[Sketch of simple closed curve counting in the orientable case]
  The proof proceeds in 3 steps.
  \begin{enumerate}[Step 1:]
  \item For any simple closed curve $\gamma$ and any $L > 0$, consider the measure $\mu_L$ on the space $\ml(\os_g)$ of measured laminations.
    \begin{align*}
      \mu_L \coloneqq \frac{1}{L^{6g-6}} \sum_{\alpha \in \mcg(\os_g)} \delta_{\frac{1}{L}\alpha \gamma}
    \end{align*}
  \item Letting $L$ go to $\infty$, $\left\{ \mu_L \right\}$ converges to measure $\mu$ that is $\mcg(\os_g)$-invariant.
    By ergodicity of the $\mcg(\os_g)$-action on $\ml(\os_g)$ with respect to the Thurston measure, we have that the limiting measure $\mu$ is a constant multiple $c$ times the Thurston measure.
  \item To show that the constant $c$ is positive, one needs to average over the moduli space $\mo(\os_g)$, using Mirzakhani's integration formula for the Weil-Petersson volume form.
  \end{enumerate}
\end{proof}

To replicate this proof in the non-orientable setting, we pick the original simple closed curve $\gamma$ to be a \emph{two-sided} curve, and replace the exponent $6g-6$ with $h + 1$, where $h$ is the critical exponent of $\mcg(\no_g)$.
With these replacements, we have that the measures $\mu_L$ converge to an $\mcg(\no_g)$-invariant measure supported on $\ml^+(\no_g)$ (this is a result of \textcite{erlandsson2023mapping}).
% Since our results imply ergodicity of the $\mcg(\no_g)$ action on $\pml^+(\no_g)$ with respect to the Patterson-Sullivan measures, we conjecture that there is an $\mcg(\no_g)$-ergodic measure supported on $\ml^+(\no_g)$, since $\ml^+(\no_g) = \pml^+(\no_g) \times \mathbb{R}$.
% If this holds, one will have completed Step 2 of the proof for non-orientable surfaces.
Since our results imply ergodicity of the $\mcg(\no_g)$ action on $\pml^+(\no_g)$ with respect to the Patterson-Sullivan measures, we can ask the following questions about the limiting measure $\mu$ as well as the product of the Patterson-Sullivan measures with the Lebesgue measure.
\begin{question}
  Is there an $\mcg(\no_g)$ ergodic measure on $\ml^+(\no_g)$ that is absolutely continuous with the product of the Patterson-Sullivan measure and the Lebesgue measure, when $\ml^+(\no_g)$ is identified with $\pml^+(\no_g) \times \mathbb{R}$?
\end{question}

\begin{question}
  Is the limiting measure $\mu$ absolutely continuous with respect to an $\mcg(\no_g)$ ergodic measure on $\ml^+(\no_g)$?
\end{question}
If the answer both the questions is yes, then one will have completed Step 2 of the proof for non-orientable surfaces.

Making Step 3 work for non-orientable surfaces is still open: this would require coming up with a recursive formula for volumes of $\systole(\no_g)$ (or some other finite volume subset), and not all of $\teich(\no_g)$, since that has infinite $\nu_N$-volume.

In the lowest complexity case, namely for $\no_{1,3}$ (i.e. the projective plane with $3$ punctures), simple closed curve counting has been established via related methods.
Gamburd, Magee, and Ronan have proved a counting result for simple closed curves by constructing a conformal measure of non-integer Hausdorff dimension on the limit set (\cite[Theorem 10]{10.4007/annals.2019.190.3.2}), and then using that conformal measure to count simple closed curves (\cite[Theorem 2]{10.1093/imrn/rny112}).

% While just mixing of the geodesic flow is not strong enough to obtain the error terms for the counting function $N(R)$, exponential mixing is strong enough, and the error terms provide a way to count simple closed curves, and obtain a version of \eqref{eq:counting-simple-closed} for non-orientable surfaces.
% \todo[inline]{State this as a question}

% \todo[inline]{Something about IETs with flips.}

% \subsubsection*{Random non-orientable hyperbolic surfaces} Since the Weil-Petersson volume of $\mo(\os_{g})$ is finite, one can sample a random orientable hyperbolic surface from this space: using the asymptotics for simple closed curves, Mirzakhani computed the expected values of various geometric properties of random hyperbolic surfaces sampled using this distribution, e.g. systole, diameter, Cheeger constant, etc.
% \todo{Maybe get a citation for this, or be more specific?}

% The analogous volume form $\nu_N$ on $\mo(\no_g)$ has infinite mass, and as a result cannot be used to sample a random non-orientable hyperbolic surface.
% However, $\systole(\no_g)$ has finite $\nu_N$-mass, and can be used to sample a random non-orientable hyperbolic surface: if we establish a version of \eqref{eq:counting-simple-closed}, we can hope to compute the expectation of geometrically meaningful random variables over this probability space.

\subsection*{Interval exchange transformations with flips}

Teichm\"uller spaces of non-orientable surfaces also show up in the context of \emph{interval exchange transformations with flips}.
The dynamics of interval exchange transformations are closely related to the dynamics of horizontal/vertical flow on an associated quadratic differential, which is related to the geodesic flow on the Teichm\"uller surface via Masur's criterion (a version of which holds in the non-orientable setting as well).
IETs with flips do not have very good recurrence properties: in fact, almost all of them (with respect to the Lebesgue measure) have a periodic point (see \cite{nogueira_1989}) and the set of minimal IETs with flips have a lower Hausdorff dimension (see \cite{skripchenko2018hausdorff}).
To understand the IETs which are uniquely ergodic, one is naturally led to determine which ``quadratic differentials'' on non-orientable surfaces are recurrent.
A necessary but not sufficient condition for recurrence of a Teichm\"uller geodesic is that its forward and backward limit points lie in the limit set.
From this perspective, Theorems \ref{thm:rational-approximation} and \ref{cor:geolimset} can
be seen as a statement about the closure of the recurrent set.
Constructing a measure supported on the closure of the recurrent set can be then used to answer questions about uniquely ergodic IETs with flips.

\subsubsection*{Geometric finiteness for mapping class subgroups}

One can think of $\mcg(\no_g)$ as a subgroup of $\mcg(\os_{g-1})$ (where $\os_{g-1}$ is the orientation double cover of $\no_g$), where the embedding is obtained by lifting mapping classes on $\no_g$ to orientation preserving mapping classes on $\os_{g-1}$.
The image of $\mcg(\no_g)$ is an infinite-index subgroup, and stabilizes an isometrically embedded copy of $\teich(\no_g)$ inside $\teich(\os_{g-1})$.

For subgroups of mapping class groups, the notion of convex-cocompactness was introduced by \textcite{farb2002convex}: these groups have good properties with respect to their dynamics on the Teichmüller space.
A natural generalization of these subgroups, inspired by the Kleinian setting, is the notion of geometric finiteness.
While there is not universally agreed upon notion of geometric finiteness for mapping class subgroups, the following two classes are subgroups are considered to be geometrically finite by any reasonable definition.

\begin{enumerate}[(i)]
\item Veech groups: These are stabilizers of Teichmüller discs in $\teich(\os_{g-1})$ which are finitely generated.
  They are lattices in $\mathrm{SL}_2(\mathbb{R})$, and their action on the Teichmüller discs they stabilize is well understood via hyperbolic geometry.
\item Combinations of Veech groups: \textcite{leininger2006combination} show that if two Veech groups $H$ and $K$ share a maximal parabolic subgroup $A$, the subgroup they generate is $H \ast_A K$ (after possibly conjugating by a pseudo-Anosov).
\end{enumerate}

The key emphasis with these two examples is that there are only finitely many cusps, i.e. finitely many conjugacy classes of reducible elements.
However, that is not the case for $\mcg(\no_g)$, it stabilizes an isometrically embedded sub-manifold, and yet there are infinitely many conjugacy classes of reducible elements.
Despite having infinitely many ``cusps'', our results show that it is still possible to do Patterson-Sullivan theory on $\mcg(\no_g)$, which is a departure from the Fuchsian/Kleinian setting, where finite Bowen-Margulis measure requires finitely many cusps.

% This suggests that for subgroups of mapping class groups, one can do Patterson-Sullivan theory even for groups that are not considered to be .
% \todo{Maybe go into more detail. Should I even be saying the geometric finiteness is too restrictive. Is that rude?}

\subsection*{Organization of the thesis}

The main results in this thesis appear in Chapters \ref{chap:limit-set-paper} and \ref{chap:stat-convex-cocompact}.
Both the chapters begin with the necessary background and notation required to state and prove the results in the chapter: some of the notation differs between the chapters since they appear as two separate papers.
In Chapter \ref{chap:future-directions}, we list two approaches we tried in order to prove our theorem, that we did not end up relying upon by the end.
These approaches are interesting in their own right, and lead to more questions about the mapping class group of non-orientable surfaces.

\subsubsection*{Organization of Chapter \ref{chap:limit-set-paper}}

Section \ref{sec:backgr-meas-foli} contains the background on non-orientable surfaces and measured foliations, and section \ref{sec:backgr-limit-sets} contains the background on limit sets of mapping class subgroups.
These sections can be skipped and later referred to if some notation or definition is unclear.
Section \ref{sec:lower-bound-limit-set} contains the proof of \autoref{thm:rational-approximation}, section \ref{sec:upper-bound-limit-set} contains the proof of Theorem \ref{cor:geolimset}, and section \ref{sec:fail-quasi-conv} contains the proof of \autoref{thm:qc-fail}.
Sections \ref{sec:lower-bound-limit-set}, \ref{sec:upper-bound-limit-set}, and \ref{sec:fail-quasi-conv} are independent of each other, and can be read in any order.

This chapter has also appeared in publication as \cite{limitsetkhan}.

\subsubsection*{Organization of Chapter \ref{chap:stat-convex-cocompact}}

In this subsection, we outline the key ideas behind the proof of the main theorems, and how they relate to each other.
Interested readers can however read the sections in any order.

\subsubsection*{Weak convexity of $\systole(\no_g)$}
We construct a projection map from $\teich(\no_g)$ to $\systole(\no_g)$ which takes any one-sided curve of length less than $\vept$ and increases its length to $\vept$, while keeping the lengths and twists of other curves constant.
We then use Minsky's product region theorem to show that this projection map increases distance by only a factor of $(1 + \vepd)$, where $\vepd$ can be picked to be arbitrarily small.

\subsubsection*{Statistical convexity of $\systole(\no_g)$}
To show that geodesics in $\systole(\no_g)$ stay away from the thin part, we construct a random walk on $\systole(\no_g)$, and compute the probability of a single step of the random walk entering the thin part, and show that this probability is small.
Estimating this probability reduces to computing an average over a ball in $\mathbb{H}$ because of Minsky's product region theorem.
The random walk argument gives us that the number of geodesics of length at most $R$ entering the thin part is at most $\exp((\hNP - 1)R)$, where $\hNP$ is the discrete analog of the volume growth entropy of $\systole(\no_g)$.
However, the total number of geodesics of length at most $R$ grows like $\exp((\hLP)R)$, where $\hLP$ is the growth rate of the number of lattice points.
To show that the probability of a geodesic entering the thin part is exponentially small, we need to relate the two entropy terms, and show that $\hLP > \hNP - 1$.

\subsubsection*{Showing $\hLP = \hNP$}
We prove entropy equality by inducting on the complexity of the surface.
We first show it for surfaces with Euler characteristic equal to $-1$ using direct methods, and reduce the inductive step to proving an estimate on complexity length for geodesic segments that spend a definite fraction of their time in thin part.

\subsubsection*{Complexity length estimate}

In this section, we
% use the machinery of complexity length introduced by \textcite{dowdall2023lattice} to
show that geodesic segments that spend a small but definite fraction of time near their end in the thin part are rare.
We do this by showing that $\hNP$ for a proper subsurface is strictly smaller than $\hNP$ for the entire surface, and use the machinery of complexity length (due to \textcite{dowdall2023lattice}), which builds upon Minsky's product region theorem and hierarchical hyperbolicity of Teichmüller space, to show that geodesic segments ending in the thin part are rare.

%%% Local Variables:
%%% TeX-master: "main"
%%% End: