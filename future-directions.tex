In this chapter, we list some stronger results that we tried to prove, but still remain open, as well as alternative approaches to some of the techniques we used.

\section{Statistical convex core of $\teich(\no_g)$}
\label{sec:stat-conv-core-of-full}

Recall that in this thesis, we showed that the action of $\mcg(\no_g)$ on $\systole(\no_g)$ is statistically convex-cocompact for arbitrarily small $\vept$.
While this is good enough to get many of the Patterson-Sullivan theoretic results, since the limit set of the $\mcg(\no_g)$ action on $\systole(\no_g)$ is the same as the limit set of $\mcg(\no_g)$ action on $\teich(\no_g)$, and $\systole(\no_g)$ is distorted by an arbitrarily small amount in $\teich(\no_g)$.
However, having the action of $\mcg(\no_g)$ on all of $\teich(\no_g)$ be statistically convex-cocompact would provide us better error terms for the counting results we obtain from Patterson-Sullivan theory.

The main issue with showing statistical convex-cocompactness for $\teich(\no_g)$ is that the random walk methods no longer work.
We outline why that is the case in Section \ref{sec:why-approach-fails}, but we reproduce the argument here for the reader's convenience.

If we wanted to make Proposition \ref{prop:rw-recurrence} work on $\teich(\no_g)$, we would need to similarly show the random walk on $\teich(\no_g)$ is recurrent in a similarly strong sense: i.e. the probability of a length $n$ trajectory staying in the thin part decays exponentially in $n$.
A consequence of this requirement is that the expected return time to the thick part is finite.

Unlike $\core(\teich(\no_g))$, $\teich(\no_g)$ has two kinds of thin regions.
\begin{itemize}
\item[-] Thin region where only two-sided curves get short.
\item[-] Thin region where some one-sided curve also gets short.
\end{itemize}

It is the second kind of thin region that poses a problem for $\teich(\no_g)$.
Minsky's product region theorem (Theorem \ref{thm:prno}) tells us that up to additive error, the metric on these thin regions looks like a product of metrics on some copies of $\mathbb{R}$ (corresponding to the one-sided short curves), some copies of $\mathbb{H}$ (corresponding to the two-sided short curves), and a Teichmüller space of lower complexity.
Since the random walk is controlled by the metric, the random walk on this product metric space is a product of random walks on each of the components.

In particular, the random walk on the $\mathbb{R}$ component is a symmetric random walk on a net in $\mathbb{R}$: i.e. a symmetric random walk on $\mathbb{Z}$.
Symmetric random walks on $\mathbb{Z}$ are known to be recurrent, but only in a weak sense: they recur to compact subsets infinitely often, but the expected return time is unbounded.

This means we cannot hope to prove exponentially decaying upper bounds on the probability that a long random walk trajectory stays in the thin part, since that would lead to finite expected return times.
This is why the random walk approach fails for $\teich(\no_g)$.

Despite the failure of the random walk methods, we still believe that the action of $\mcg(\no_g)$ on $\teich(\no_g)$ is statistically convex-cocompact.
To see why that might be true, we consider our examples of geodesic segments that leave $\systole(\no_g)$: they are coarsely of the form $[p, \gamma p]$, where $\gamma$ is a psuedo-Anosov on a subsurface where some boundary component is one-sided.
As a result of that, the boundary component gets short, and the geodesic segments leaves $\systole(\no_g)$.
However, we have seen that the number of mapping classes has a lower exponential growth rate than the exponential growth rate of the entire mapping class group (this is the content of Section \ref{sec:entr-gap-cons}).
If we can show that all the geodesic segments that leave $\systole(\no_g)$ are of this form, we will have established statistical convex-cocompactness for $\teich(\no_g)$.

\section{Upgrading random walk phenonomena to uniform measure}
\label{sec:upgr-rand-walk}

Recall that in order to prove equality of net point entropy and lattice point (i.e. the contents of Sections \ref{sec:equal-latt-point} and \ref{sec:line-gap-compl}), we needed to show that the proportion of bad points (see Definition \ref{defn:bad-points}) goes to $0$ with respect to the uniform measure on a ball of radius $R$ as $R$ goes to $\infty$.
If one replaces the requirement that the net points be sampled from the uniform measure on a ball of radius $R$, and instead allow them to sampled from an $n$-fold convolution of a finitely supported measure (i.e. the random walk measure associated to an $n$-step random walk), the result is easier to prove.

In fact, in the random walk measure setting, the following result of Taylor and Sisto holds.
\begin{theorem}[Theorem 1.1 of \cite{taylorsisto}]
  Let $\mu$ be a finitely supported measure on $\mcg(\os)$, where $\os$ is a surface of finite type.
  Let $w_n$ be the random walk on $\mcg(\os)$ driven by the measure $\mu$.
  Then there exists a constant $C$ such that the following holds with high probability.
  \begin{align*}
    \frac{\log n}{C} \leq \sup_{V \sqsubset \os} d_{\mathcal{C}(V)}(1, w_n) \leq C \log n
  \end{align*}
\end{theorem}

Inspired by the above result, we can make a similar statement (and pose a question) about net points with respect to the uniform measure on a ball of radius $R$.
\begin{question}
  \label{ques:reduce-to-uniform}
  Let $w_R$ be a net point picked uniformly at random from a ball of radius $R$.
  Does there exist a constant $C > 0$ such that the following holds with high probability?
  \begin{align}
    \label{ques:log-bound}
    \frac{\log n}{C} \leq \sup_{V \sqsubset \os} d_{\mathcal{C}(V)}(1, w_n) \leq C \log n
  \end{align}
\end{question}

How does the above question relate to bad points?
Recall that for a geodesic segment joining a bad point that is within distance $R$ of the base point $p$ to the base point spends at least $\vepb R$ time in the thin part of $\teich(\os)$.
If a geodesic segment spends $\vepb R$ time in the thin part, then its subsurface projection to some subsurface grows faster than $\log R$, i.e. it grows linearly in $R$.

If we can answer a version of Question \ref{ques:reduce-to-uniform}, we can prove the results of Section \ref{sec:equal-latt-point} without reducing to the complexity length arguments in Section \ref{sec:line-gap-compl}.

More generally, it is easier to prove statistical results with respect to $n$-fold random walk measures, as opposed to uniform measures, since we can exploit independence between the steps of the random walk.
One can in such situations ask whether a statement made with respect to the random walk measure continues to hold with respect to the uniform measure.
An example of such a random walk measure to uniform measure upgrade appears in \textcite{choi2022pseudoanosovs}.
They show that in a ball of radius $R$ in the mapping class group (with respect to the word metric), the proportion of elements that are not pseudo-Anosov goes exponentially decays to $0$, by showing a similar result holds when counting with respect to the $n$-fold convolution of the random walk generating measure, and transferring the estimate (with a worse exponential decay constant) to the uniform measure case.

Performing such a transformation from random walk phenomena to uniform measure phenomena in a very general setting might be quite hard, since it is known that the hitting measure of the random walk on the boundary is mutually singular with respect to the limiting measure of the uniform measure, at least in the orientable case (see \textcite{gadre2015word}).



%%% Local Variables:
%%% TeX-master: "main"
%%% End: